\documentclass[../main.tex]{subfiles}
 
\begin{document}

\subsection{Zitieren und Verweisen}
\label{zitieren_und_verweisen}
Hier wird Abbildung \ref{fig:img_cat} referenziert.

Zieren einer Quelle: "Penis Penis", (\cite[Seite 11]{theoretischeinformatik}).

Weil ich es kann\footcite[][]{spektumperformance}, verweise ich auf eine Quelle. % Die beiden leeren [][] in den Fußnoten-Zitaten werden benötigt, damit das nicht wiederholte Anzeigen von Zitaten in den Fußnoten korrekt funktioniert

Um es nicht noch einmal einzubinden, hier nochmal der Link zu Kapitel \ref{kapitel1} für das Katzenbild. Übrigens kann auch ein Unterkapitel, wie \ref{zeichnungen_und_formeln} verwiesen werden.

\subsection{Zeichnungen und Formeln}
\label{zeichnungen_und_formeln}
Als Beispiel für Zeichnungen und Formeln hier die formale Definition eines Automaten:
$$A_{alterAutomat} = (\Sigma, S, \delta, s_{0}, F)$$
$$\Sigma = \left\{0, 1, 2\right\}$$
$$S = \left\{Blau, BlauB, Rot1, Rot2, Rot3, RotB\right\}$$
$$\delta(Blau, 0) = Blau; \delta(Blau, 1) = Rot1; \delta(Blau, 2) = RotB;$$
$$\delta(BlauB, 0) = Blau; \delta(BlauB, 1) = Rot1; \delta(BlauB, 2) = RotB;$$
$$\delta(Rot1, 0) = BlauB; \delta(Rot1, 1) = Rot2; \delta(Rot1, 2) = RotB;$$
$$\delta(Rot2, 0) = BlauB; \delta(Rot2, 1) = Blau3; \delta(Rot2, 2) = RotB;$$
$$\delta(Rot3, 0) = BlauB; \delta(Rot3, 1) = RotB; \delta(Rot3, 2) = RotB;$$
$$\delta(RotB, 0) = BlauB; \delta(RotB, 1) = RotB; \delta(RotB, 2) = RotB$$
$$s_{0} = Blau$$
$$F = \left\{\right\}$$
Und die dazugehörige grafische Darstellung des Automaten. Diese ist ausgelagert, um die Datei übersichtlich zu halten:

\subfile{drawings/automat.tex}

\end{document}