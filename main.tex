% Anforderungen für Transferleistungen (Stand 03/2019):
% % Umfang Auftragsklärung: 2000 Zeichen (Richtwert)
% % Umfang Selbstreflexion: 2000 Zeichen (Richtwert)
% % Umfang Transferleistung: 10 Textseiten +/- 10% (ohne Gliederung, Verzeichnisse, Anhänge, Abbildungen)
% % Schriftgröße: Times New Roman (12pt) oder Arial (11pt)
% % Ränder: Alle 2cm
% % Zeilenabstand: 1,25 mit Blocksatz und Silbentrennung
% % Zitation: übliche Zitierweise aus dem ausgewählten Fachgebiet

\documentclass[12pt, a4paper, onecolumn, oneside, titlepage, ngerman]{article}

\usepackage[main=ngerman, english]{babel} % Setzt Dokument auf Deutsch

\usepackage[a4paper, left=2cm, right=2cm, top=2cm, bottom=2cm]{geometry} % Setzt alle Seiteränder auf 2cm
\usepackage{subfiles} % Für mehr Übersicht werden die Kapitel in Subfiles ausgelagert
\usepackage[T1]{fontenc} % Font wird westeuropäisch angefordert
\usepackage[utf8]{inputenc} % Zeichensatz auf UTF8
\usepackage{csquotes} % Aktiviert Zitate
\MakeOuterQuote{"} % Aktiviert die Erkennung von Anführungszeichen, sodass diese sprachenspezifisch (für Deutsch, weil oben definiert) dargestellt werden
\usepackage{hyperref} % Aktiviert automatische Erkennung von Hyperlinks
\usepackage{tikz} % Wird für die Zeichnungen direkt in LaTeX benötigt
\usepackage{float} % Für korrekte Positionierung von Abbildungen
\usepackage{graphicx} % Um Bilder einbinden zu können

\usepackage{microtype} % Wenn man dem Internet glaubt, macht dieses Paket allgemein einfach alles besser, also rein damit!

\usepackage{pgfplots} % Für Diagramme und sowas
\pgfplotsset{compat=1.15} % Setzt Version
 
% Aktiviert Zeilenabstand von 1.25
\usepackage{setspace}
\setstretch{1.25}
\usepackage{mathptmx} % Aktiviert Times New Roman als Standardschrift

\newcounter{savepage} % Aktiviert die Möglichkeit der Fortsetzung von Seitenzahlen zu einem späteren Zeitpunkt

% Aktiviert Quellenverzeichnis mit APA-Style und legt Zitatstil für Fußleiste fest
\usepackage[backend=biber, bibencoding=utf8, style=apa, citestyle=authortitle]{biblatex}
% Füge zu Fußzeilen-Verweisen die Jahreszahl hinzu
\usepackage{xpatch}
\xapptobibmacro{cite}{\setunit{\nametitledelim}\printfield{year}}{}{}

\addbibresource{quellen.bib} % Bindet Quellen aus quellen.bib ein
% \nocite{*} % Zeigt alle (auch nicht zitierten) Einträge der Quellendatei im Literaturverzeichnis an

% Aktiviert die Kommaseparation beim Verwenden mehrerer Fußnotenzitate direkt nacheinander
\usepackage{fnpct}
\AdaptNoteOpt\footcite\multfootcite

\begin{document}

\pagenumbering{Roman} % Zu Anfang sollen römische Seitenzahlen angezeigt werden

\subfile{Titelseite.tex} % Titelseite einbinden

\newpage

\pagestyle{headings} % Es sollen Seitenzahlen mit der aktuellen Überschrift angezeigt werden

\tableofcontents % Inhaltsverzeichnis einbinden
\setcounter{tocdepth}{1} % Sections als oberste Ebene des Inhaltsverzeichnis festlegen

\newpage

% Abbildungsverzeichnis
\addcontentsline{toc}{section}{\listfigurename} % Hinzufügen zum Inhaltsverzeichnis
\listoffigures

\newpage

% Tabellenverzeichnis
\addcontentsline{toc}{section}{\listtablename} % Hinzufügen zum Inhaltsverzeichnis
\listoftables

\newpage

\setcounter{savepage}{\arabic{page}} % Vorige Seitenzahl speichern
\pagenumbering{arabic} % Es wird mit Arabischen Zahlen die Seitenzahl gezählt

\section{Kapitel 1}
\subfile{sections/Kapitel1.tex}
\label{kapitel1}

\section{Kapitel 2}
\subfile{sections/Kapitel2.tex}
\label{kapitel2}

\newpage % Auf neuer Seite wird das Literaturverzeichnis ausgegeben 

\pagenumbering{Roman} % Das Literaturverzeichnis wird römisch nummeriert
\setcounter{page}{\thesavepage} % Die römische Nummerierung soll fortgesetzt werden

\printbibliography[heading=bibintoc] % Literaturverzeichnis

\end{document}
